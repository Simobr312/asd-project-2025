\documentclass[a4paper, 11pt]{article}
\usepackage[utf8]{inputenc}
\usepackage[T1]{fontenc}
\usepackage[italian]{babel}
\usepackage{amsmath}
\usepackage{amssymb}
\usepackage{amsfonts}
\usepackage{geometry}
\geometry{a4paper, top=3cm, bottom=3cm, left=2.5cm, right=2.5cm}
\usepackage{graphicx}
\usepackage{hyperref}
\hypersetup{
    colorlinks=true,
    linkcolor=blue,
    filecolor=magenta,      
    urlcolor=cyan,
    pdftitle={Report Progetto Reti Bayesiane},
    pdfpagemode=FullScreen,
}
\usepackage{listings}
\usepackage{xcolor}

% Definizione di uno stile per il codice C++
\definecolor{codegreen}{rgb}{0,0.6,0}
\definecolor{codegray}{rgb}{0.5,0.5,0.5}
\definecolor{codepurple}{rgb}{0.58,0,0.82}
\definecolor{backcolour}{rgb}{0.95,0.95,0.92}

\lstdefinestyle{cppstyle}{
    backgroundcolor=\color{backcolour},   
    commentstyle=\color{codegreen},
    keywordstyle=\color{magenta},
    numberstyle=\tiny\color{codegray},
    stringstyle=\color{codepurple},
    basicstyle=\ttfamily\footnotesize,
    breakatwhitespace=false,         
    breaklines=true,                 
    captionpos=b,                    
    keepspaces=true,                 
    numbers=left,                    
    numbersep=5pt,                  
    showspaces=false,                
    showstringspaces=false,
    showtabs=false,                  
    tabsize=2,
    language=C++
}
\lstset{style=cppstyle}

\title{Report sull'Algoritmo di Variable Elimination\\per Reti Bayesiane}
\author{Simone Riccio}
\date{\today}

\begin{document}

\maketitle

\begin{abstract}
Questo documento descrive l'implementazione dell'algoritmo di Variable Elimination per il calcolo delle probabilità marginali in una rete bayesiana. L'obiettivo è fornire una panoramica dell'algoritmo, delle scelte implementative in C++ e di un esempio pratico di utilizzo.
\end{abstract}

\tableofcontents
\newpage

\section{Introduzione}
Le reti bayesiane sono modelli grafici probabilistici che rappresentano un insieme di variabili aleatorie e le loro dipendenze condizionali tramite un grafo aciclico diretto (DAG). Il calcolo di inferenza, come la determinazione della probabilità marginale di una variabile data un'evidenza, è un compito fondamentale ma computazionalmente complesso.

\section{L'Algoritmo di Variable Elimination}
L'algoritmo di Variable Elimination (VE) è una tecnica per calcolare esattamente le probabilità marginali in una rete bayesiana. L'idea centrale è di eliminare le variabili non di interesse una alla volta, sommando (o marginalizzando) su di esse.

\subsection{Fattori}
L'algoritmo opera su una struttura dati chiamata "fattore". Un fattore $\phi(X_1, \dots, X_k)$ è una funzione che mappa ogni possibile assegnazione di valori alle variabili $X_1, \dots, X_k$ a un numero reale non negativo. Le tabelle di probabilità condizionale (CPT) della rete sono i fattori iniziali.

\subsection{Passi dell'Algoritmo}
L'algoritmo si articola nei seguenti passi:
\begin{enumerate}
    \item \textbf{Inizializzazione:} Per ogni variabile, si crea un fattore corrispondente alla sua CPT.
    \item \textbf{Riduzione dell'evidenza:} Se sono presenti variabili di evidenza, si riducono i fattori per riflettere i valori osservati.
    \item \textbf{Eliminazione delle variabili:} Per ogni variabile da eliminare (che non è né di query né di evidenza):
    \begin{enumerate}
        \item Si raccolgono tutti i fattori che includono la variabile.
        \item Si calcola il prodotto di questi fattori.
        \item Si marginalizza la variabile dal fattore prodotto, creando un nuovo fattore.
    \end{enumerate}
    \item \textbf{Risultato finale:} Dopo aver eliminato tutte le variabili necessarie, si calcola il prodotto dei fattori rimanenti.
    \item \textbf{Normalizzazione:} Il risultato viene normalizzato per ottenere una distribuzione di probabilità.
\end{enumerate}

\section{Implementazione in C++}
La nostra implementazione in C++ si concentra sulla creazione di una classe \texttt{Factor} flessibile e sull'orchestrazione dei passi dell'algoritmo VE.

\subsection{Struttura Dati per i Fattori}
Un fattore è stato implementato come una classe che contiene:
\begin{itemize}
    \item Un elenco delle variabili coinvolte.
    \item Una mappa o un vettore per memorizzare i valori del fattore per ogni assegnazione.
\end{itemize}

\begin{lstlisting}[caption={Esempio di scheletro della classe Factor in C++.}, label=lst:factor]
class Factor {
public:
    // Costruttori, metodi per il prodotto, la marginalizzazione, etc.

private:
    std::vector<std::string> variables;
    std::map<std::vector<int>, double> values;
};
\end{lstlisting}

\section{Esempio di Esecuzione}
Consideriamo una semplice rete bayesiana (es. la rete "Sprinkler"). Qui si può inserire un'immagine della rete.

% \begin{figure}[h!]
%     \centering
%     \includegraphics[width=0.5\textwidth]{path/to/your/image.png}
%     \caption{Esempio di rete bayesiana.}
%     \label{fig:bn_example}
% \end{figure}

Si mostra l'output del programma per una query specifica, ad esempio $P(\text{WetGrass} | \text{Sprinkler}=\text{true})$.

\section{Conclusioni}
L'algoritmo di Variable Elimination si è dimostrato efficace per il calcolo di inferenza esatta. L'implementazione in C++ ha richiesto un'attenta gestione delle strutture dati, in particolare per la rappresentazione e la manipolazione dei fattori.

\end{document}